\documentclass[12pt,a4paper]{article}
\usepackage[T1]{fontenc}
\usepackage[latin2]{inputenc}
\usepackage[english,polish]{babel}
% \pagestyle{empty}
\title{Tytu"l artyku"lu}
\author{Adam Kowalski\thanks{E-mail: adam@host1.domena1}}
\date{10 marca 2001}
\begin{document}
\maketitle

\begin{abstract}
Tre"s"c abstraktu. 
Szkielet artyku"lu polskiego.
\end{abstract}

% \tableofcontents
% \listoffigures
% \listoftables

\section{Wst"ep}

\section{Rozwini"ecie}

W pracy \cite{label1} podano ciekawe fakty.

\section{Zako"nczenie}

\appendix

\section{Tytu"l pierwszego dodatku}

\begin{thebibliography}{99}

\bibitem{label1}
Praca nr 1.

\end{thebibliography}

\end{document}
