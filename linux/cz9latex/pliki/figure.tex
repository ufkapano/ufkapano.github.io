% Rozne rzeczy zwiazane z rysunkami.
\documentclass[12pt]{article}
% Pakiet potrzebny do wlaczania rysunkow w EPS
\usepackage{graphicx}
% Mozna zdefiniowac wlasne jednostki dlugosci:
\newlength{\krok}
\setlength{\krok}{10pt}
\begin{document}

% \listoffigures

W srodowisku {\em picture} dozwolone jest tylko polecenie {\em put}.

W srodowisku {\em picture} jednostka dlugosci jest {\em unitlength}
o domyslnej dlugosci 1pt. Mozna to zmienic poza tym srodowiskiem
poleceniem {\em setlength}.

\begin{figure}
\begin{center}
\begin{picture}(200,100)       % rozmiar w pt
\put(0,0){\framebox(200,100)[t]{Ramka}}
\put(10,10){Napis}
\end{picture}
\end{center}
\caption[Opcjonalny krotki opis rysunku.]{
\label{fig1}
Tutaj dlugi lub krotki opis rysunku.
W Phys. Rev. podpis jest pod rysunkiem.}
\end{figure}

\begin{figure}
\begin{center}
\setlength{\unitlength}{10pt}  % zmiana domyslnej jednostki dlugosci
\begin{picture}(20,10)
\put(0,0){\framebox(20,10)[t]{Ramka}}
\put(1,1){Napis}
\end{picture}
\end{center}
\caption{
\label{fig2}
Rysunek ze zmieniona domyslna jednostka dlugosci na 10pt.
Wszystkie wymiary i wspolrzedne podzielilem przez 10.}
\end{figure}

\begin{figure}
\begin{center}
\begin{picture}(297,210)
\put(0,0){\framebox(297,210)[t]{FIGURE (unit=pt)}}
\put(100,10){\dashbox{10}(100,50)[bl]{Box 100x50}}
\put(160,150){Point(160,150)}
\put(150,150){\makebox(0,0){o}}
\put(0,0){\line(2,5){60}}
\put(60,150){\circle{40}}
\put(60,150){\vector(3,1){20}}
\put(150,100){\oval(80,40)}
\put(200,60){\oval(80,40)[bl]}
\multiput(10,10)(5,10){8}{\makebox(0,0){x}}
\end{picture}
\end{center}
\caption[Krotki opis do rysunku.]{
\label{fig3}
Dlugi podpis pod rysunkiem.}
\end{figure}

\begin{figure}
\begin{center}
\begin{picture}(387,273)       % rozmiar w pt
%\put(0,0){\framebox(387,273)[t]{}}
\includegraphics{y_vs_x.eps}
\end{picture}
\end{center}
\caption[Opcjonalny krotki opis rysunku.]{
\label{fig4}
Tutaj dlugi lub krotki opis rysunku.}
\end{figure}


\end{document}
