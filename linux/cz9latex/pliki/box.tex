% Rozne rzeczy zwiazane z ramkami i odstepami.
\documentclass[12pt]{article}
%
% Zmiany w domyslnych rozmiarach strony.
\setlength{\topmargin}{-0.5in}    % 1in - 0.5in
\setlength{\oddsidemargin}{-0.5in}  % 1in - 0.5in
\setlength{\evensidemargin}{-0.5in} 
% To ma znaczenie tylko przy druku dwustronnym
%
% Zmiana rozmiaru korpusu strony.
\setlength{\textheight}{10in}
\setlength{\textwidth}{6in}
%
% Zmiana domyslnego wciecia na poczatku kazdego akapitu.
% Jednorazowo robi sie to przez polecenie \noindent.
% Rozpoczecie akapitu to pusta linia lub polecenie \par.
\setlength{\parindent}{12mm}
%
% Zmiana domyslnego pionowego odstepu pomiedzy akapitami.
\setlength{\parskip}{10mm}
\begin{document}

Wersje skrocone polecen nie majace parametrow opcjonalnych:
\fbox{tekst}, \mbox{tekst} (pewna calosc nie dzielona przez LaTeX-a.

Rozne mozliwosci ramek (podobnie dla makebox)
\begin{itemize}
\item \framebox[10cm][l]{na lewo}
\item \framebox[10cm][r]{na prawo}
\item \framebox[10cm][c]{w srodku}
\item \framebox[10cm][s]{kilka roznych wyrazow}
\item \framebox[3\width]{potrojna}
\end{itemize}


\end{document}
