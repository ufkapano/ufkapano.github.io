% Rozne rzeczy zwiazane z czcionkami.
\documentclass[12pt]{article}
\usepackage[english,polish]{babel}
\begin{document}

Jest {\em deklaracja em} i \emph{polecenie emph}
do wyr"o"rnienia tekstu.
\begin{em}
Mo"rna te"r zrobi"c jakby "srodowisko em (begin-end).
\end{em}

Na definicje stylu czcionki sk"ladaj"a si"e trzy elementy:
kr"oj (family), seria (series) i odmiana (shape).

\begin{itemize}
\item Deklaracje kroju:
{\rmfamily antykwa},
{\sffamily czcionka bezszeryfowa},
{\ttfamily czcionka maszyny do pisania}.
\item Deklaracje serii:
{\bfseries pismo po"lgrube},
{\mdseries pismo normalnej grubo"sci}.
\item Deklaracje odmiany:
{\itshape kursywa},
{\scshape kapitaliki},
{\slshape pismo pochy"le},
{\upshape pismo proste}.
\end{itemize}

\begin{itemize}
\item Polecenia kroju:
\textrm{antykwa},
\textsf{czcionka bezszeryfowa},
\texttt{czcionka maszyny do pisania}.
\item Polecenia serii:
\textbf{pismo po"lgrube},
\textmd{pismo normalnej grubo"sci}.
\item Polecenia odmiany:
\textit{kursywa},
\textsc{kapitaliki},
\textsl{pismo pochy"le},
\textup{pismo proste}.
\end{itemize}

\end{document}
