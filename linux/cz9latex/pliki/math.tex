% Rozne rzeczy zwiazane z trybem math.
\documentclass[12pt]{article}
\begin{document}

Tu zmieniam czcionke na rm, trzeba tez dac odstepy
(jest jednak zalecenie, zeby unikac polecenia {\rm }):
\begin{displaymath}
a\in B\ {\rm and}\ b\in B.
\end{displaymath}

Zamiast displaymath mozna o wiele krocej
\[ a\in B\ {\rm and}\ b\in B. \]

Tu stosuje polecenie mbox (ten sam efekt i jest lepsze):
\begin{displaymath}
a\in B\ \mbox{and}\ b\in B.
\end{displaymath}

\begin{displaymath}
{\frac{Licznik}{Mianownik}}
\end{displaymath}

Uzycie wykrzyknika zapobiega duzemu odstepowi po symbolu calki:
\begin{displaymath}
\int {dx} f(x), \int\! {dy} g(y).
\end{displaymath}

Aby znak mnozenia pojawial sie tylko przy zlamaniu linii wzoru
trzeba zastosowac backslash: $a \* b$.

W pismie nawias krecony stalej wielkosci uzyskujemy przez
$\{\, x: x \le 1 \, \}$.
Ponizej definiowanie przedzialow funkcji:
\begin{equation}
f(x) =
\left{
\begin{array}{ll}
+\infty   & \mbox{for} \ x<1,  \\
-\epsilon & \mbox{for} \ 1<x<2,  \\
0         & \mbox{for} \ 2<x.
\end{array}
\right. 
\end{equation}


\end{document}
